%%%%%%%%%%%%%%%%%%%%%%%%%%%%%%%%%%%%%%%%%%%%%%%%%%%%%%%%%%%%%%%%%%
% ACM ICPC 2016-2017, NEERC                                      %
% Northern Subregional Contest                                   %
% St Petersburg, October 22, 2016                                %
%%%%%%%%%%%%%%%%%%%%%%%%%%%%%%%%%%%%%%%%%%%%%%%%%%%%%%%%%%%%%%%%%%
% Problem J. Java2016                                            %
%                                                                %
% Original idea         Georgiy Korneev                          %
% Problem statement     Georgiy Korneev                          %
% Test set              Georgiy Korneev                          %
%%%%%%%%%%%%%%%%%%%%%%%%%%%%%%%%%%%%%%%%%%%%%%%%%%%%%%%%%%%%%%%%%%
% Problem statement                                              %
%                                                                %
% Author                Georgiy Korneev                          %
%%%%%%%%%%%%%%%%%%%%%%%%%%%%%%%%%%%%%%%%%%%%%%%%%%%%%%%%%%%%%%%%%%

\begin{problem}{Java2016}{java2016.in}{java2016.out}{\timeLimit}

% Original idea : Georgiy Korneev
% Text          : Georgiy Korneev
% Tests         : Georgiy Korneev

John likes to learn esoteric programming languages.
Recently he discovered the probabilistic programming language Java2K.
Built-in functions of Java2K have only a certain probability 
to do whatever you intend them to do.

The Java2K programming is very hard, so John designed a much simpler
language for training: Java2016. Built-in operators of Java2016 are
deterministic, while their operands are random. 
Each value in Java2016 is a positive integer in the range $0..255$, inclusive.

Java2016 supports six operators of three precedencies:
\vspace{-0.8ex}\begin{bnf}\begin{eqnarray*}
<expression>&::=& <expression> \chr{min} <sum> | <expression> \chr{max} <sum> | <sum>\\
<sum>       &::=& <sum> \chr{+} <term> | <sum> \chr{-} <term> | <term> \\
<term>      &::=& <term> \chr{*} <factor> | <term> \chr{/} <factor> | <factor> \\
<factor>    &::=& \chr{(} <expression> \chr{)} | \chr{?} | <macro>
\end{eqnarray*}\end{bnf}\vspace{-2em}

Minimum (\chr{min}) and maximum (\chr{max}) operators are defined as usual.
Addition (\chr{+}), subtraction (\chr{-}) and multiplication (\chr{*}) are
defined modulo 256.
The result of the division (\chr{/}) is rounded towards zero.
If the divider is zero, the program crashes.
The argument of the operator is a result of another operator,
evenly distributed random value (\chr{?}), or macro substitution.

For instance, the probability that \txt{?/?/?} is evaluated to zero is $98.2\%$,
while the probability of the crash~is~$0.8\%$.

The Java2016 program consists of zero or more macro definitions, followed
by the resulting expression. Each macro definition has a form of 
\vspace{-0.8ex}\begin{bnf}\begin{eqnarray*}
<macrodef>  &::=& <macro> \chr{=} <expression> \\
<macro>     &::=& \chr{a} \dots \chr{z}
\end{eqnarray*}\end{bnf}\vspace{-2em}

The macro should be defined before the first use. It may not be redefined.
The macro is expanded to its definition on each use. For instance,
\\\verb|    a = ?  max ?|
\\\verb|    (a max a) / a|
\\is expanded to \txt{((?~~max~?)~max~(?~~max~?))~/~(?~~max~?)}.

John is going to add probabilistic constants to Java2016, so for each
possible constant value he needs a program that successfully
evaluates to this value with at least one-half probability.
Crashes are counted toward failures. 

\InputFile

The input contains a single integer $c$~--- the target constant
($0 \le c \le 255$).  

\OutputFile

Output a Java2016 program that successfully evaluates to constant $c$ with probability
no less than $1/2$. The total length of the program should not exceed 
$256$ characters (excluding spaces).

\Examples

\begin{example}
\exmp{
0
}{
? /?/ ?
}%
\exmp{
1
}{
a = ?  max ?
(a max a) / a
}%
\end{example}%
\end{problem}

